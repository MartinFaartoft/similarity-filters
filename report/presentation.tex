% Author: Till Tantau
% Source: The PGF/TikZ manual
\documentclass[a4paper,11pt]{article}
\usepackage[utf8]{inputenc}
\usepackage{listings}
\usepackage{amsmath}    % need for subequations
\usepackage{graphicx}   % need for figures
\usepackage{verbatim}   % useful for program listings
\usepackage{color}      % use if color is used in text
%\usepackage{subfigure}  % use for side-by-side figures
\usepackage{hyperref}   % use for hypertext links, including those to external documents and URLs
\usepackage{url}
\usepackage{float}
\usepackage{todonotes}
\usepackage{tikz}
\usepackage{enumitem}
\usepackage{hyperref}
\usepackage{pdfpages}
\usepackage{caption}
\usepackage{epsfig}
\usepackage{subcaption}
\usepackage{listings}
\usepackage{color}
\usepackage{amsfonts}
\usepackage{latexsym}
\usepackage[T1]{fontenc} % use for allowing < and > in cleartext
\usepackage{fixltx2e}    % use for textsubscript
\usepackage[linesnumbered,boxed,ruled]{algorithm2e}
% \newcommand{\BigO}[1]{\ensuremath{\operatorname{O}\left(#1\right)}}
\newcommand{\BigO}[1]{\ensuremath{\mathop{}\mathopen{}\mathcal{O}\mathopen{}\left(#1\right)}}

\definecolor{mygreen}{rgb}{0,0.6,0}
\definecolor{mygray}{rgb}{0.5,0.5,0.5}
\definecolor{mymauve}{rgb}{0.58,0,0.82}
\lstset{ %
  backgroundcolor=\color{white},   % choose the background color; you must add \usepackage{color} or \usepackage{xcolor}
  basicstyle=\footnotesize,        % the size of the fonts that are used for the code
  breakatwhitespace=false,         % sets if automatic breaks should only happen at whitespace
  breaklines=true,                 % sets automatic line breaking
  captionpos=b,                    % sets the caption-position to bottom
  commentstyle=\color{mygreen},    % comment style
  deletekeywords={...},            % if you want to delete keywords from the given language
  escapeinside={\%*}{*)},          % if you want to add LaTeX within your code
  extendedchars=true,              % lets you use non-ASCII characters; for 8-bits encodings only, does not work with UTF-8
  %frame=single,                    % adds a frame around the code
  keepspaces=true,                 % keeps spaces in text, useful for keeping indentation of code (possibly needs columns=flexible)
  keywordstyle=\color{blue},       % keyword style
  language=Octave,                 % the language of the code
  morekeywords={*,...},            % if you want to add more keywords to the set
  numbers=left,                    % where to put the line-numbers; possible values are (none, left, right)
  numbersep=5pt,                   % how far the line-numbers are from the code
  numberstyle=\tiny\color{mygray}, % the style that is used for the line-numbers
  rulecolor=\color{black},         % if not set, the frame-color may be changed on line-breaks within not-black text (e.g. comments (green here))
  showspaces=false,                % show spaces everywhere adding particular underscores; it overrides 'showstringspaces'
  showstringspaces=false,          % underline spaces within strings only
  showtabs=false,                  % show tabs within strings adding particular underscores
  stepnumber=2,                    % the step between two line-numbers. If it's 1, each line will be numbered
  stringstyle=\color{mymauve},     % string literal style
  tabsize=2,                       % sets default tabsize to 2 spaces
  %title=\lstname                   % show the filename of files included with \lstinputlisting; also try caption instead of title
}

\bibliographystyle{plain}
\begin{document}
\graphicspath{ {./images/} }
\date{April 22nd 2014}
\title{Similarity Filters\\An Introduction}

\author{Marcus Gregersen\\
\texttt{mabg@itu.dk}
\and Martin Faartoft\\
\texttt{mlfa@itu.dk}
\and Rick Marker\\
\texttt{rdam@itu.dk}}
\clearpage\maketitle
\thispagestyle{empty}
\setcounter{page}{1}

\section{Introduction}
In the following we investigate the Similarity Search problem. Different approaches have been explored by earlier work, and in this paper we will investigate two of those; 'Distance-Sensitive Bloom Filters'\cite{paper:harvard} by Kirsch and Mitzenmacher, and 'Locality-Sensitive Bloom Filter for Approximate Membership Query'\cite{paper:hua} by Hua et al.

\subsection{The Problem}
The Similarity Search problem can be stated as follows: Given a set of elements $S$, determine if there exists an element $s \in S$ that is 'close' to a given query element $q$. Where 'close' is defined as being within a given distance, $d$ using a certain metric.

% Begræns problemet
In the following we consider the Similarity Search problem for bit-vectors of length $l$, and Hamming distance as metric. This reduction maintains a high degree of generality, since many domains can be encoded with bit-vectors.

\subsection{Definitions}
\begin{description}
\item \textbf{Hamming distance}

A distance metric for bit-vectors. The Hamming distance between two vectors $v_1, v_2$ is defined as the number of positions $i$ where $v_1[i] \neq v_2[i]$.

\item \textbf{Bloom Filter}

\begin{quotation}
A Bloom filter is an inexact representation of a set that allows for false positives when queried; that is, it can sometimes say that an element is in the set when it is not. In return, a Bloom filter offers very compact storage: less than 10 bits per element are required for a 1\% false positive probability, independent of the size or number of elements in the set.\cite{paper:bloom}
\end{quotation}


\item \textbf{Locality-Sensitive Hashing}

Regular hashing tries to spread out the hash-values of different elements, to minimize the probability of a collision. Locality-Sensitive Hashing (LSH) tries to group similar elements, by maximizing the collision probability for similar elements.

The LSH is closely tied to the distance metric, and many distance metrics have no known LSH. The Hamming distance metric on bit-vectors has a particularly simple LSH: Sample a fixed number of bits from the input vector, uniformly at random. 

It is intuitively obvious, that if two elements only differ on the non-sampled bits, then they will hash to the same value, and thus be considered 'close' by the LSH.

The more bits two vectors have in common, the higher the probability will be that a random LSH will hash them to the same value.

% \item \textbf{$\epsilon$-closeness}
%   \begin{itemize}
%     \item If an element $s \in S$ differs from $x$ in at most an $\epsilon$-fraction of the bits, it is said to be $\epsilon$-close to $x$. The data structure must return “close”.
%     \item If every vector $s \in S$ differs from $x$ in at least a $\delta$-fraction of the bits, then the data structure should return 'not close'.
%     \item In all other cases, i.e. when the distance to the nearest vector is between $\epsilon$ and $\delta$, the data structure can give any answer it likes.
%   \end{itemize}

\end{description}

\subsection{Naïve Approaches}

\paragraph{Brute force}
The most obvious idea for solving the Similarity Search problem, is by brute force. Store the elements $S$ in a linked list. When a query is made, simply scan the linked list, and calculate the distance from each element $s \in S$ the query element $q$. If $s$ satisfies the distance requirement, a match has been found. If the end of the linked list is reached, no match exists.

This will give a correct answer, and will work well for small problem instances, But the linear requirement on time and space in the total number of bits in $S$, is prohibitively expensive for many real-world applications.

\paragraph{Bloom filters} If we relax the requirements, to allow false positives, we can change these characteristics to support either constant-time queries, or space-efficiency, but not both. 

While keeping in mind that standard Bloom filters only answer exact membership, we see that one way of achieving this is by relying on extra insertions. Everytime an element is added to the Bloom filter, all elements within distance $d$ are generated and added in addition to the original element. 

Another way to go around this is by sacrificing time. This can be done by still doing normal insertions but instead querying for all elements with distance $d$ of the element we are comparing to. This approach has the benefit of still being space efficient but will cause an exponential blowup of the running time.

\section{Related work} % Ideer fra related works

\subsection{Distance-Sensitive Bloom Filters}
To be able to perform non-exact matching in Bloom filters, Kirsch and Mitzenmacher propose, in \cite{paper:harvard}, a novel way of using Bloom filters. They replace the ordinary hash functions with LSH hash functions. When a query is received, they ask all the hash functions if they have seen something similar to this element, and if more than a certain threshold, $t$ returns true, then the data structure returns true, meaning that a similar element was found in the data structure.

The main disadvantage to this approach, is that it introduces the possibility of false negatives. 

% Langt fra sikker på hvor specifikt det skal være. Det her lyder MEGET vagt, men ved ikke om det giver mening at give præcise angivelser.
%In their article they perform 2 experiments, varying different factors, both showing that the more space used gives fewer false positives and false negatives. The "false positive/size" and "false negative/size" ratios they achieve, although not impressive, Kirsch and Mitzenmacher note that the ratios can not be improved unless they take into account special properties of the elements in the filter.
In their experiments they manage to achieve false postive rates and false negative rates of 1.5\% and 0.2\% respectively for 1000 elements, and 0.016\% and 0.003\% for 1000. For 1000 elements they use 64\% of the total space of the elements, and for 10000 elements they use 51\% of the total space.
% Hvordan adskiller det sig fra normal bloom filters, hvor du har et threshold, t

\subsection{"Locality-Sensitive Bloom Filter for Approximate Membership Query"}
The paper written by Hua et al.\cite{paper:hua}, takes a slightly different approach than \cite{paper:harvard}. They use a standard Bloom Filter data structure, with LSH functions in place of ordinary hash functions, but no thresholding. They call this a 'Locality Sensitive Bloom Filter'. This 'naive' approach, has a high probability of both false positives and false negatives. To minimize these, they augment the Bloom filter with additional data structures, that will not be discussed further here.

In the report they have chosen to use a 'proximity measure', which makes it impossible to compare the results to the ones found in \cite{paper:harvard} directly and all data presented is only available through graphs. From the graphs, however, we can see that they achieve between 85\% and 100\% accuracy.
% Helt ærlig, hvordan fanden skal det lort læses?

\section{Ideas and Preliminary Results}
We have mainly been focused on the 'Distance-Sensitive Bloom Filters'\cite{paper:harvard} paper, and have achieved the following:
\\

We have implemented the data-structure, and can roughly reproduce their results.
\\

We suggest a small improvement to the way the LSH functions choose what bits to sample. If we force each bit to be sampled the same number of times, we make the thresholding more robust against 'unfortunate' choices of bits to sample. 
\\

Building on the previous idea, we suggest a way to eliminate false negatives completely: If every bit is only sampled a fixed number of times $n$, and a query element $q$ is 'close', meaning it only differs from s in at most $\epsilon$ bits, then setting the threshold to: $t = k - n \cdot \epsilon$ will eliminate false negatives. The downside is an increase in the number of false positives, and the fact that the data structure no longer works for some parameter settings ($t < 1$).
\\

Finally we aim to experimentally investigate if the choice of threshold in \cite{paper:harvard} can be improved.

\begin{thebibliography}{}

\bibitem{paper:harvard}
A. Kirsch and M. Mitzenmacher, "Distance-Sensitive Bloom Filters", Proc. Eighth Workshop Algorithm Eng. and Experiments (ALENEX), 2006.

\bibitem{paper:bloom}
Bonomi, Flavio and Mitzenmacher, Michael and Panigrahy, Rina and Singh, Sushil and Varghese, George, "An Improved Construction for Counting Bloom Filters", Algorithms – ESA 2006 

\bibitem{paper:hua}
Yu Hua, Bin Xiao, Bharadwaj Veeravalli, Dan Feng, "Locality-Sensitive Bloom Filter for Approximate Membership Query", IEEE Transactions on Computers, Vol. 61, No. 6, 2012
 
\end{thebibliography}

\end{document}
